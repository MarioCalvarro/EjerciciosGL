\documentclass[10pt,a4paper,openright]{book}

%% Formateo del título del documento
\title{Soluciones de los Ejercicios. Geometría lineal}
\author{Mario Calvarro Marines}
\date{}

%% Formateo del estilo de escritura y de la pagina
\pagestyle{plain}
\setlength{\parskip}{0.35cm} %edicion de espaciado
\setlength{\parindent}{0cm} %edicion de sangría
\clubpenalty=10000 %líneas viudas NO
\widowpenalty=10000 %líneas viudas NO
\usepackage[top=2.5cm, bottom=2.5cm, left=3cm, right=3cm]{geometry} % para establecer las medidas de los margenes
\usepackage[spanish]{babel} %Para que el idioma por defecto sea español
\usepackage{ulem} % para poder subrayar entornos especiales como las secciones

%% Texto matematico y simbolos especiales
\usepackage{amsmath} %Paquetes para mates
\usepackage{amsfonts} %Paquetes para mates
\usepackage{amssymb} %Paquetes para mates
\usepackage{stmaryrd} % paquete para mates
\usepackage{latexsym} %Paquetes para mates
\usepackage{cancel} %Paquete tachar cosas

%% Ruta de las fotos e inclusion de las mismas
\usepackage{graphicx}
\graphicspath{{./fotos/}}

%% Inclusion de referencias cruzadas por defecto y específicas
\usepackage{hyperref}

%% Paquete para definir y utilizar colores por el documento
\usepackage[dvipsnames,usenames]{xcolor} %activar e incluir colores
    %% definicion de los colores que se van a utilizar en cada cabecera
    \definecolor{capitulos}{RGB}{60,0,0}% gama de colores de los capitulos
    \definecolor{secciones}{RGB}{95,8,5}% gama de colores de las secciones
    \definecolor{subsecciones}{RGB}{140,36,31}% gama de colores de las subsections
    \definecolor{subsubsecciones}{RGB}{188,109,79}% gama de colores de las subsubsections
    \definecolor{teoremas}{RGB}{164,56,32}% gama de colores para los teoremas
    \definecolor{demos}{RGB}{105,105,105} % gama de colores para el cuerpo de las demostraciones

%% Paquete para la edición y el formateo de capítulos, secciones...
\usepackage[explicit]{titlesec}
    %% Definición del estilo de los capítulos, secciones, etc...
    \titleformat{\chapter}[display]{\normalfont\huge\bfseries\color{capitulos}}{}{0pt}{\Huge #1}[\titlerule]
    \titleformat{\section}{\normalfont\Large\bfseries\color{secciones}}{}{0pt}{#1}
    \titleformat{\subsection}{\normalfont\large\bfseries\color{subsecciones}}{}{0pt}{\uline{#1}}
    \titleformat{\subsubsection}{\normalfont\normalsize\bfseries\color{subsubsecciones}}{}{0pt}{#1}

%% Paquete para el formateo de entornos del proyecto
\usepackage{ntheorem}[thmmarks]
    %% Definicion del aspecto de los entornos matematicos del proyecto
    \theoremstyle{break}
    \theoremheaderfont{\normalfont\bfseries\color{teoremas}}
    \theorembodyfont{\itshape}
    \theoremseparator{\vspace{0.2cm}}
    \theorempreskip{\topsep}
    \theorempostskip{\topsep}
    \theoremindent0cm
    \theoremnumbering{arabic}
    \theoremsymbol{}
    \theoremprework{\vspace{0.2cm} \hrule}
    \theorempostwork{\vspace{0.2cm}\hrule}
        \newtheorem*{defi}{Definición}

    \theoremprework{\vspace{0.25cm}}
        \newtheorem*{theo}{Teorema}

    \theoremprework{\vspace{0.25cm}}
    	\newtheorem*{coro}{Corolario}

    \theoremprework{\vspace{0.25cm}}
    	\newtheorem*{lema}{Lema}

    \theoremprework{\vspace{0.25cm}}
    	\newtheorem*{prop}{Proposición}

    \theoremheaderfont{\normalfont}
    \theorembodyfont{\normalfont\color{demos}}
    \theoremsymbol{\hfill\square}
    	\newtheorem*{demo}{\underline{Demostración}:}

    \theoremheaderfont{\normalfont}
    \theorembodyfont{\normalfont}
    	\newtheorem*{obs}{\underline{Observación}:}
    	\newtheorem*{ej}{\underline{Ejemplo}:}

%% Definicion de operadores especiales para simplificar la escritura matematica
\DeclareMathOperator{\dom}{dom}
\DeclareMathOperator{\img}{img}
\DeclareMathOperator{\rot}{rot}
\DeclareMathOperator{\divg}{div}
\newcommand{\dif}[1]{\ d#1}

%% Paquete e instrucciones para la generacion de los dibujos
\usepackage{pgfplots}
\pgfplotsset{compat=1.17}
\usepackage{tkz-fct}
\usepgfplotslibrary{fillbetween}
\usepackage{tikz,tikz-3dplot}
\tdplotsetmaincoords{80}{45}
\tdplotsetrotatedcoords{-90}{180}{-90}
\usetikzlibrary{arrows}
    %% style for surfaces
    \tikzset{surface/.style={draw=blue!70!black, fill=blue!40!white, fill opacity=.6}}

    %% macros to draw back and front of cones
    %% optional first argument is styling; others are z, radius, side offset (in degrees)
    \newcommand{\coneback}[4][]{
        %% start at the correct point on the circle, draw the arc, then draw to the origin of the diagram, then close the path
        \draw[canvas is xy plane at z=#2, #1] (45-#4:#3) arc (45-#4:225+#4:#3) -- (O) --cycle;
    }
    \newcommand{\conefront}[4][]{
        \draw[canvas is xy plane at z=#2, #1] (45-#4:#3) arc (45-#4:-135+#4:#3) -- (O) --cycle;
    }
    
    \tikzset{middlearrow/.style={decoration={markings, mark= at position 0.5 with {\arrow{#1}},},postaction={decorate}}}
    
    \usetikzlibrary{decorations.markings}
    
    \newcommand{\AxisRotator}[1][rotate=0]{
    \tikz [x=0.25cm,y=0.60cm,line width=.2ex,-stealth,#1] \draw (0,0) arc (-150:150:1 and 1);
    }
    
    \usetikzlibrary{shapes}


\begin{document}
\maketitle
\setcounter{tocdepth}{3}% para que salgan las subsubsecciones en el indice
\tableofcontents
\chapter{Hoja 1}%
\label{cha:hoja_1}
\section{Ejercicio 5}%
\label{sec:5}
Sea $c_i = \left( 0, \ldots, 0, 1, 0, \ldots, 0 \right): i \in \{1, \ldots, n\}$. Como referencia $\mathcal{R}_c = \{\left( 0, \ldots, 0 \right), c_1, \ldots, c_n\}$ y $c_0 := \left( 0, \ldots, 0 \right)$. 

Definimos $f: \mathbb{A}^{n}_k \rightarrow \mathbb{A}^{n}_k$, como $f\left( c_i \right) = c_i,\ \forall i$ y $f\left( 0, \ldots, 0 \right) = \left( a_1, \ldots, a_n \right) \in \mathbb{A}^{n}_k,\ a_1, \ldots, a_n \in K$.

Dado $x \in \mathbb{A}^{n}_k,\ \exists x_1, \ldots, x_n \in K: x = c_0 + x_1\overrightarrow{c_0c_1} + \ldots + x_n \overrightarrow{c_0c_n} \Rightarrow$
\[
f\left( x \right) = f\left( c_0 \right)  + \overrightarrow{f} \left( \sum_{i=1}^{n} x_i \overrightarrow{c_0c_i} \right) = \left( a_1, \ldots, a_n \right) + \sum_{i=1}^{n} x_i \overrightarrow{f} \left( \overrightarrow{c_0 c_i} \right) = (*) 
\]
Como:
\[
c_i = \left( 0, \ldots, 0 \right) + \overrightarrow{c_0c_i},\ c_i = f\left( c_i \right) = f\left( 0, \ldots, 0 \right) + \overrightarrow{f} \left( \overrightarrow{c_0c_i} \right) = \left( a_1, \ldots, a_n \right) + \overrightarrow{f} \left( \overrightarrow{c_0c_i} \right) \Rightarrow
\]\[
\overrightarrow{f} \left( \overrightarrow{c_0c_i} \right) = c_i - \left( a_1, \ldots, a_n \right) = \left( -a_1, \ldots, 1 - a_i, \ldots, -a_n \right) \in k^n
\]
Entonces,
\[
(*) = \left( a_1, \ldots, a_n \right) + \sum_{i=1}^{n} x_i\left( -a_1, \ldots, 1 - a_i, \ldots, -a_n \right) = \left( a_1, \ldots, a_n \right) + x_1 \left( 1 - a_1, \ldots, -a_n \right) + \ldots +  
\]\[
+ x_n \left( -a_1, \ldots, 1 - a_n \right) = \left( a_1 + x_1 - a_1x_1 - \ldots -a_1x_n, \ldots, a_n - a_nx_1 - \ldots - a_n x_n + x_n \right) = 
\]\[
= \left( a_1\left( 1 - x_1 - \ldots - x_n \right) + x_1, \ldots, a_n \left( 1 - x_1 - \ldots - x_n \right) + x_n \right)
\]

\section{Ejercicio 7}%
\label{sec:7}
Sean las rectas cuya intersección buscamos:
\begin{gather*}
\begin{cases}
    L_1 = \{X_0 - X_1 - X_2 = 0\} \\
    L_2 = \{2X_0 + X_1 - 2X_2 = 0\}     
\end{cases} \subset \mathbb{P}^{n}_{k} 
.\end{gather*}
Pasamos al dual: 
\begin{gather*}
    L_1 \leftrightarrow \left( 1 : -1 : -1 \right) \in \mathbb{P}^{2}_{k} \\
    L_2 \leftrightarrow \left( 2 : 1 : -2 \right) \in \mathbb{P}^{2}_{k} 
.\end{gather*}
Por tanto, la intersección es: 
\[
    0 = \begin{vmatrix} U_0 & U_1 & U_2\\ 1 & -1 & -1\\ 2 & 1 & -2 \end{vmatrix} = 3U_0 + 3U_2
\]
Esta ecuación se corresponde con una recta en el dual, es decir, un punto en el proyectivo:
\[
L_3^* = \{3U_0 + 3U_2 = 0\} \leftrightarrow \left( 1 : 0 : 1 \right) \in \mathbb{P}^{2}_{k} = L_1 \cap L_2
\]
Haciendo la intersección con el punto original nos queda:
\[
    0 = \begin{vmatrix} X_0 & X_1 & X_2\\ 2 & 1 & -1\\ 1 & 0 & 1 \end{vmatrix} = X_0 - 3X_1 - X_2 \Rightarrow \{X_0 - 3X_1 - X_2 = 0\} \subset \mathbb{P}^{2}_{k} 
\]

\section{Ejercicio 8}%
\label{sec:8}
Recta que pasa por $\left( 1 : -1 : -1 \right), \left( 2 : 1 : -2 \right)$ en el proyectivo:
\[
    0 = \begin{vmatrix} X_0 & X_1 & X_2\\ 1 & -1 & -1\\ 2 & 1 & -2 \end{vmatrix} = 3X_0 + 3X_2
\]
Intersección con $2X_0 + X_1 - X_2 = 0$. Pasamos al dual: ecuación generada por los puntos $\left( 3 : 0 : 3 \right)$ y $\left( 2 : 1 : -1 \right)$
\[
    0 = \begin{vmatrix} U_0 & U_1 & U_2\\ 3 & 0 & 3\\ 2 & 1 & -1 \end{vmatrix} = -3U_0 + 9U_1 + 3U_2
\]
los coeficientes coinciden con las coordenadas del punto intersección en el proyectivo:
\[
\left( -3 : 9 : 3 \right) = \left( -1 : 3 : 1 \right) \in \mathbb{P}^{2}_{k} 
\]

\section{Ejercicio 9}%
\label{sec:ejercicio_9}
Construimos la recta $L$ que pasa por $A$ y $C$:
\[
    \begin{vmatrix} X_0 & X_1 & X_2\\2 & 0 & 3\\ 3 & 1 & 0 \end{vmatrix} = 0 \Leftrightarrow L := \{-3X_0 + 9X_1 + 2X_2 = 0\} 
\]
Vemos que $B \not\in L$ ya que $-3\left( -1 \right) + 9\left( 1 \right) + 2\left( 2 \right) \neq 0 \Rightarrow$ \underline{NO} están alineados.

\section{Ejercicio 10}%
\label{sec:ejercicio_10}
Pasamos al dual:
\[
\begin{cases}
    X_0 - X_1 + 2X_2 = 0\\
    3X_0 + 2X_1 - X_2 = 0
\end{cases} \Rightarrow \begin{cases}
    \left( 1 : -1 : 2 \right)\\
    \left( 3 : 2 : -1 \right) 
\end{cases}  
\]\[
0 = \begin{vmatrix} U_0 & U_1 & U_2\\ 1 & -1 & 2\\ 3 & 2 & -1 \end{vmatrix} = \boxed{-3U_0 + 7U_1 + 5 U_2 = 0} 
\]

\section{Ejercicio 11}%
\label{sec:ejercicio_11}
Calculemos las intersecciones de los siguientes pares de rectas haciendo uso del dual:
\[
L_1 = \{X_0 - X_1 + 2X_2 = 0\},\ L_2 = \{3X_0 + 2X_1 - X_2 = 0\} \Rightarrow
\begin{cases}    
L_1 \leftrightarrow \left( 1 : -1 : 2 \right) \in \mathbb{P}^{2^*}_{k}\\
L_2 \leftrightarrow \left( 3 : 2: -1 \right) \in \mathbb{P}^{2^*}_{k} 
\end{cases} 
\]\[
0 = \begin{vmatrix} U_0 & U_1 & U_2\\ 1 & -1 & 2\\ 3 & 2 & -1 \end{vmatrix} = -3U_0 + 7U_1 + 5U_2 \rightarrow \left( -3 : 7 : 5 \right) \in \mathbb{P}^{2}_{k} 
\]
que será el punto de intersección $L_1 \cap L_2$.

Por otro lado,
\[
L_3 = \{3X_0 - 2X_1 - X_2 = 0\},\ L_4 = \{2X_0 + 2X_1 + X_2 = 0\} \Rightarrow \begin{cases}
    L_3 \leftrightarrow \left( 3 : -2 : -1 \right)\\
    L_4 \leftrightarrow \left( 2 : 2 : 1 \right) 
\end{cases} 
\]\[
0 = \begin{vmatrix} U_0 & U_1 & U_2\\ 3 & -2 & -1\\ 2 & 2 & 1 \end{vmatrix} = -5U_1 + 10U_2 \rightarrow \left( 0 : -1 : 2 \right) \in \mathbb{P}^{2}_{k} 
\]
que será el punto de intersección $L_3 \cap L_4$.

Por último, recta que pasa por los dos puntos de intersección $\left( -3 : 7 : 5 \right),\ \left( 0 : -1 : 2 \right)$:
\[
    0 = \begin{vmatrix} X_0 & X_1 & X_2\\ -3 & 7 & 5\\ 0 & -1 & 2 \end{vmatrix} = \boxed{19X_0 + 6X_1 + 3X_2 =  0} 
\]

Podemos hacerlo de otra manera, parametrizamos el haz de rectas de un par y buscamos la recta que pasa por el otro punto de intersección:
\[
\left( 0 : -1 : 2 \right) \in \{t_0 \left( X_0 - X_1 + 2X_2 \right) + t_1\left( 3X_0 + 2X_1 - X_2 \right) = 0\} 
\]\[
5t_0 - 4t_1 = 0 \Leftrightarrow \left( t_0 : t_1 \right) = \left( 4 : 5 \right) \in \mathbb{P}^{1}_{k} 
\]
Y sustituimos $\left( 4, 5 \right)$ en $\left( t_0, t_1 \right)$.

\section{Ejercicio 12}%
\label{sec:ejercicio_12}
\begin{enumerate}
    \item[a)] $Y = X^2 - X + 2$.
    \[
    X_0^2\left[ \left( \frac{X_1}{X_0} \right)^2 - \frac{X_1}{X_0} - \frac{X_2}{X_0} + 2 \right] = 0 \rightarrow X_1^2 - X_1X_0 - X_2X_0 + 2X_0^2 = 0
    \]\[
    X_0 = 0 \Rightarrow X_1^2 = 0 \Rightarrow X_1 = 0 \Rightarrow \left( 0 : 0 : 1 \right) 
    \]
    Como solo hay un punto en el infinito podemos clasificar la curva como parábola.

    \item[b)] $X^2 - Y^2 = 1$
    \[
    X_1^2 - X_2^2 - X_0^2 = 0 \rightarrow^{X_0 = 0} X_1^2 - X_2^2 = 0 \Leftrightarrow X_1^2 = X_2^2 \Leftrightarrow X_1 = \pm X_2 
    \]
    Puntos en el infinito: $\left( 0 : 1 : -1 \right),\ \left( 0 : 1 : 1 \right)$.

    \item[c)] $X^2 + XY + Y^2 = 1$
    
    Homogeneizamos: $X_1^2 + X_1X_2 + X_2^2 = 0$

    Buscamos soluciones en $\mathbb{R}$.
    \[
    X_1^2 + X_1X_2 + X_2^2 = 0 \Leftrightarrow X_1 = X_2 = 0 \Rightarrow \left( 0 : 0 : 0 \right) \not\in \mathbb{P}^{2}_{k} 
    \]
    por lo que no es punto del infinito.

    Buscamos soluciones en $\mathbb{C}$: 
    \[
    X_1 \neq 0, X_2 \neq 0
    \]\[
    \frac{X_1}{X_2} + 1 + \frac{X_2}{X_1} = 0 \Rightarrow s + 1 + \frac{1}{s} = 0 \Rightarrow s^2 + s + 1 = 0 \Rightarrow s = \frac{-1 \pm \sqrt{1 - 4}}{2} = \frac{-1 \pm \sqrt{3}i}{2} \Rightarrow \Rightarrow
    \]\[
    X_1 = \frac{-1 \pm \sqrt{3}i}{2} X_2 \Rightarrow \begin{cases}
        \left( 0 : \frac{-1 + \sqrt{3}i}{2} : 1 \right) \\
        \left( 0 : \frac{-1 - \sqrt{3}i}{2} : 1 \right) 
    \end{cases} \text{ Puntos en el infinito.}  
    \]
    \item[d)] $Y = X^3$

    Homogeneizamos: $X_0^2X_1 = X_1^3$

    Puntos en el infinito $\Rightarrow X_0 = 0$
    \[
    \Rightarrow X_1^3 = 0, X_2 \text{ libre} \Rightarrow \text{Tomamos } X_2 = 1
    \]\[
    \left( 0 : 0 : 1 \right) \text{ punto en el infinito.}  
    \]
    \item[e)] $Y^2 = X^3$

    Homogeneizamos: $X_0X_2^2 = X_1^3$

    Punto en el infinito: $X_0 = 0 \Rightarrow X_1^3 = 0$
    \[
    \left( 0 : 0 : 1 \right) \text{ punto en el infinito.} 
    \]
    Esta curva es la misma que la anterior en el proyectivo.
\end{enumerate}

\chapter{Hoja 2}%
\label{cha:hoja_2}
\section{Ejercicio 1}%
\label{sec:ejercicio_1}
\[
\begin{cases}
    X_0 - X_2 - X_3 = 0\\
    X_1 - 2X_2 + X_3 = 0
\end{cases} \Rightarrow X_0 - X_2 - X_3 + \lambda\left( X_1 - 2X_2 + X_3 \right) = 0
\]
contiene a la recta dada.

Si pasa por $\left( 0 : 1 : 1 : 0 \right) \Rightarrow 0 - 1 - 0 + \lambda\left( 1 - 2 \cdot 1 + 0 \right) = 0 \Leftrightarrow -1 - \lambda = 0 \Leftrightarrow \lambda = -1$.

El plano buscado es $X_0 - X_2 - X_3 - \left( X_1 - 2X_2 + X_3 \right) = \ldots = X_0 - X_1 + X_2 - 2X_3 = 0$.

\section{Ejercicio 2}%
\label{sec:ejercicio_2}
\[
    \Pi_1 := 0 = \begin{vmatrix} X_0 & X_1 & X_2 & X_3\\ 0 & 1 & 1 & 1\\ 1 & 0 & 1 & 0\\ 1 & 1 & 0 & 0 \end{vmatrix} = \ldots = X_0 + X_1 - X_2 = 0.
\]\[
    \Pi_1 := 0 = \begin{vmatrix} X_0 & X_1 & X_2 & X_3\\ 1 & 0 & 0 & 0\\ 0 & 2 & 1 & 1\\ 1 & 1 & -1 & 0 \end{vmatrix} = \ldots = -X_1 - X_2 + 3X_3 = 0
\]
Fijamos un valor tal que $X_3 = 1,\ X_0 = \lambda \Rightarrow$
\[
\left( \lambda : \frac{-\lambda + 3}{2} : \frac{\lambda + 3}{2} : 1 \right) 
\]
Damos valores a $\lambda$:
\begin{itemize}
    \item Si $\lambda = 0$: 
    \[
    \left( 0 : \frac{3}{2} : \frac{3}{2} : 1 \right)  
    \]
    \item Si $\lambda = 1$: 
    \[
    \left( 1 : 1 : 2 : 1 \right) 
    \]
\end{itemize}
Por tanto,
\begin{align*}
    \mathbb{P}^{1}_{k} &\rightarrow \mathbb{P} \left( W \right)\\
    \left( t_0, t_1 \right) &\mapsto \left( \underbrace{t_0 \cdot 0 + t_1\cdot 1}_{= x_0}  : \underbrace{t_0 \cdot \frac{3}{2} + t_1 \cdot 1}_{= x_1}  : \underbrace{t_0 \cdot \frac{3}{2} + t_1 \cdot 2}_{= x_2}  : \underbrace{t_0 + t_1}_{= x_3} \right)
.\end{align*}

\section{Ejercicio 3}%
\label{sec:ejercicio_3}
\textbf{\underline{Disclaimer}: No he entendido nada, posiblemente este mal copiado.} 

Tomamos $\mathbb{P}^{3}_{k}$ como $k^4 / \sim$. 

Identificamos $r \rightarrow \pi \subset K^4\ \{v_0, v_1\}$ y $r' \rightarrow \{v_2, v_3\}$ que será una base porque si no fuesen linealmente idependientes (las siguientes ecuaciones) el determinante de los coeficientes sería $0$: 
\[
r : \begin{cases}
    \sum a_ix_i = 0\\
    \sum b_i x_i = 0
\end{cases} 
\]\[
r' : \begin{cases}
    \sum c_i x_i = 0\\
    \sum d_i x_i = 0
\end{cases} \Leftrightarrow \det \begin{pmatrix} a_i\\ b_i\\ c_i\\ d_i \end{pmatrix} = 0 \Leftrightarrow \begin{pmatrix} a_i\\ b_i\\ c_i\\ d_i \end{pmatrix} \begin{pmatrix} x_0\\ \ldots\\ x_j? \end{pmatrix} = 0
\]
Tenemos?
\[
\hat{p} = \underbrace{\alpha_0 v_0 + \alpha_1 v_1}_{\omega_0} + \underbrace{\alpha_2 v_2 + \alpha_3 v_3}_{= \omega_1} 
\]\[
l = \alpha\left( \omega_0, \omega_1 \right) \Rightarrow \hat{p} = \omega_0 + \omega_1
\]

\end{document}
